\makeatletter
\makeatother
\documentclass[../main.tex]{subfiles}

\graphicspath{
    {img}
	{../img/}
}

\begin{document}
\section{Уравнения колебаний в пространстве.}
%Глава 2 Параграф 9 па Казлоўскай
$$\Delta u = \frac{1}{a^2}\frac{\partial^2u}{\partial f^2}$$

$$\delta u =\frac{\partial^2u}{\partial x^2} + \frac{\partial^2u}{\partial y^2} + \frac{\partial^2u}{\partial z^2},\ \ -\infty < x,y,z < +\infty$$
$$\delta u =\frac{\partial^2u}{\partial x^2} + \frac{\partial^2u}{\partial y^2},\ \ -\infty < x,y < +\infty$$
$$t>0, u|_{t=0}=\phi(M), \pderiv{u}{t}|_{t=0}=\psi(M), M(x,y,z)$$
Построим частное решение однородного уравнения, обладающего центральной симметричностью относительно точки $M_0$.
$$u(M,t)=u(r,t), r = r_{M,M_0}$$
$v(r,t)=ru(r,t)$ - будет удовлетворять одном. волновому уравнению.
\\ \\
Запишем оператор Лапласа в сферических координатах:
$$\Delta u = \frac{1}{r^2}\pderiv{}{r}(r^2\pderiv{u}{r})+\frac{1}{r^2sin(\theta)}\pderiv{}{\theta}(sin(\theta)\pderiv{u}{\theta})+\frac{1}{r^2sin(\theta)}\frac{\partial^2v}{\psi^2}$$

$$\Delta u = \frac{1}{r^2}\pderiv{}{r}(r^2\pderiv{u}{r})=\frac{\partial^2u}{r^2}+\frac{2}{r}\pderiv{u}{r}=\frac{1}{r}\frac{\partial^2}{\partial r^2}(ru)$$

$$\pderiv{}{r}(ru)=u+r\pderiv{u}{r}$$
$$\pderiv{^2}{r^2}(ru)=\pderiv{u}{r}+r\pderiv{^2u}{r^2}$$
$$\frac{1}{r}\pderiv{^2}{r^2}(ru)=\frac{1}{a^2}\pderiv{^2u}{t^2}-\frac{1}{r}\frac{1}{a^2}\pderiv{^2}{t^2}(ru)$$

Введем замену $v = r u $.
$$\pderivt{v}{r} = \frac{1}{a^2} \pderivt{v}{t} $$

Тогда из $u |_{t=0} = \varphi(r)$, $\pderivt{u}{t} |_{t=0} = \psi(r)$ получаем
$$v |_{t=0} = r \varphi(r)$$
$$\pderiv{v}{t}|_{t=0} = r \psi(r)$$

Если $u$ ограничена в нуле, тогда $v(0, t)=0$.

Решим одномерное волновое уравнение:

$$v(r, t) = f_1\left(t - \frac{r}{a}\right) + f_2\left(t+ \frac{r}{a}\right)$$
$$u(r, t) = \frac{1}{r} f_1 \left(t - \frac{r}{a}\right) + \frac{1}{r} f_2 \left(t+\frac{r}{a}\right)$$
\begin{itemize}
    \item \textbf{Расходящаяся сферическая волна}
    $$u_1 = \frac{1}{r} f_1\left(t - \frac{r}{a}\right)$$
    \item \textbf{Сходящаяся сферическая волна}
    $$u_2 = \frac{1}{r} f_2\left(t+\frac{r}{a}\right)$$
\end{itemize}
Решением будет сумма сферических волн
$$v(0, t) =0$$
$$f_1(t) + f_2(t) = 0 \Rightarrow f_2(t) = -f_1(t) = f(t)$$

$$u(r, t) = \frac{1}{r}f\left(t+\frac{r}{a}\right) - \frac{1}{r} f\left(t-\frac{r}{a}\right)  \xrightarrow[r \to 0]{} \frac{2}{a} f'(t)$$

$$\boxed{u(0, t) = \frac{2}{a} f'(t)} \label{eq:2.9.1}$$


\end{document}
