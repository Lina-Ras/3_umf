\makeatletter
\makeatother
\documentclass[../main.tex]{subfiles}

\graphicspath{
    {img}
	{../img/}
}

\begin{document}
\section{Метод Дюамеля для решения задачи Коши для неоднородного волнового уравнения}

\begin{equation}
    \label{eq:2.6.1}
    \frac{1}{a^2}\pderiv{^2u}{t^2} = \pderiv{^2u}{x^2} + f(x, t), -\infty \leq x \leq +\infty, t > 0
\end{equation}
\begin{equation}
    \label{eq:2.6.2}
    u|_{t=0} = \phi(x), -\infty \leq x \leq +\infty
\end{equation}
\begin{equation}
    \label{eq:2.6.3}
    \pderiv{u}{t}|_{t=0}=\psi(x), -\infty \leq x \leq +\infty
\end{equation}

Для решение задач \eqref{eq:2.6.1}-\eqref{eq:2.6.3} построим вспомогательную задачу: 
\begin{equation}
    \label{eq:2.6.4}
    \frac{1}{a^2}\pderiv{^2w_ff}{t^2}=\pderiv{^2w_f}{x^2}, -\infty \leq x \leq +\infty, t > \tau,
\end{equation}
\begin{equation}
    \label{eq:2.6.5}
    w_f(x, \tau, \tau) = 0, t = \tau
\end{equation}
\begin{equation}
    \label{eq:2.6.6}
    \pderiv{w_f}{t}(x, \tau, \tau) = f(x, \tau), -\infty \leq x \leq +\infty
\end{equation}
Откуда 
$$
    w_f(x, t, \tau)=w_f(x, t-\tau,\tau) = \frac{1}{2a}\int_{x-a(t-\tau)}^{x+a(t-\tau)}f(\xi, \tau)d\xi
$$
Если рассмотрим \eqref{eq:2.6.1}-\eqref{eq:2.6.3} для однородного уравнени, то решение можно записать как:
\begin{equation}
    u(x, t)=\pderiv{w_\phi(x, t, 0)}{t} + w_\psi(x, t, 0)
\end{equation}

$w_\phi(x, t, 0)$ - решение задачи \eqref{eq:2.6.4}-\eqref{eq:2.6.6}, 
если $f = \phi$ и $\tau = 0$.
$w_phi(x, t, 0) = \frac{1}{2a}\int_{x-at}^{x+at}\phi(\xi)d\xi$. 
$w_psi(x, t, 0) = \frac{1}{2a}\int_{x-at}^{x+at}\psi(\xi)d\xi$
Тогда 
\begin{equation}
    u(x, t) = \frac{\phi(x+at) + \phi(x - at)}{2} + \frac{1}{2a}\int_{x - at}{x+at}\psi(\xi)d\xi
\end{equation}

\begin{theorem}
    Решение задачи  \eqref{eq:2.6.1} -  \eqref{eq:2.6.3} с однородными начальными условиями представляется в виде
    \begin{equation}
        u(x, t) = a^2\int_0^tw_f(x, t, \tau)d\tau
    \end{equation}
\end{theorem}
\begin{proof}
   $$ 
   \pderiv{u}{t}=a^2w_f(x, t, t) + a^2 \int_0^t\pderiv{w_f(x,t,\tau)}{t}d\tau
   $$
   Но из задачи для $w_f$ следует, что $w_f(x, t, t) = 0$. Тогда получаем, что 
   $ \pderiv{u}{t} = a^2\int_0^t\pderiv{w_f(x, t, \tau)}{t}d\tau$
   $$
   \pderiv{^2u}{t^2}=a^2\pderiv{w_f(x, t, t)}{t} + a^2\int_0^t\pderiv{^2w_f(x, t, \tau)}{t^2}d\tau =
   a^2f(x, t) + a^2\int_0^t\pderiv{^2w_f(x, t, \tau)}{t^2}d\tau
   $$
   $$
   \pderiv{^2u}{x^2} = a^2\int_0^t\pderiv{^2w_f(x, t, \tau)}{x^2}d\tau
   $$
   Учитывая \eqref{eq:2.6.4}:
   $$
   \pderiv{^2u}{x^2} = \int_0^t\pderiv{^2w_f(x, t, \tau)}{t^2}d\tau =  a^2\int_0^t\pderiv{^2w_f(x, t, \tau)}{x^2}d\tau
   $$

\end{proof}
Решение задачи \eqref{eq:2.6.1} - \eqref{eq:2.6.3}:
$$
u(x, t) = \pderiv{w_\phi(x, t, 0)}{t} + w_f(x, t, 0) + a^2\int_0^tw_f(x, t, \tau)d\tau
$$
$$
w_f(x, t, \tau)=\frac{1}{2a}\int_{x - a(t-\tau)}^{x+a(t-\tau)}f(\xi, \tau)d\xi
$$
$$
u(x, t) = \frac{\phi(x + at) + \phi(x - at)}{2} + \frac{1}{2a}\int_{x - at}^{x + at}\psi(\xi)d\xi + \frac{a}{2}\int_0^{t}\int_{x - a(t - \tau)}^{x + a(t - \tau)}f(\xi,\tau)d\xi d\tau
$$

\end{document}