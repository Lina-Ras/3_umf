\makeatletter
\makeatother
\documentclass[../main.tex]{subfiles}

\graphicspath{
    {img}
	{../img/}
}

\begin{document}
\section{Исключение младших производных в уравнениях 2-го порядка}
Рассмотрим следующие уравнения
{\bf
\begin{enumerate}
	\item Гиперболического типа
	      \begin{equation} \label{eq:1.10.1}
			\frac{\partial ^2 u}{\partial x \partial y} + a \pderiv{u}{x} + b \pderiv{u}{y} + cu = f(x, y)
	      \end{equation}
	\item Параболического типа
	      \begin{equation} \label{eq:1.10.2}
			\frac{\partial ^2 u}{\partial y^2} + a \pderiv{u}{x} + b \pderiv{u}{y} + cu = f(x, y)
	      \end{equation}
	\item Эллиптического типа
	      \begin{equation} \label{eq:1.10.3}
		      \frac{\partial ^2 u}{\partial x^2} + \frac{\partial^2 u}{\partial y^2} + a \pderiv{u}{x} + b \pderiv{u}{y} + cu = f(x, y)
	      \end{equation}
	\item Уравнение с $n$ независимыми переменными
	      \begin{equation} \label{eq:1.10.4}
		      \sum_{i=1}^n a_i \frac{\partial^2 u }{\partial x_i^2} + \sum_{i=1}^n \lambda_i \pderiv{u}{x_i} + cu = f(x), a_i \in \{0, -1, 1\}
	      \end{equation}
\end{enumerate}}
Можно проделать дальнейшее упрощение и исключить младшие прозводные, по которым есть старшие.
В уравнениях (\ref{eq:1.10.2}) - (\ref{eq:1.10.4}) можно ввести замену
$$u(x) = v(x)e^{-\frac{1}{2} \sum \limits_{i=1, a_i \neq 0}^n \frac{\lambda_i}{a_i} x_i}$$
Для (\ref{eq:1.10.1}):
$$u(x,y) = v(x,y) e ^ {\alpha x + \beta y}$$
$$\pderiv{u}{x} = (\pderiv{v}{x}+\alpha v)A, \quad A=e ^ {\alpha x + \beta y}$$
$$\pderiv{u}{y} = (\pderiv{v}{y}+\beta v)A$$
$$\frac{\partial^2 u}{\partial x \partial y} = (\frac{\partial^2 v}{\partial x \partial y} + \beta \pderiv{v}{x} + \alpha \pderiv{v}{y} + \alpha \beta v)A$$
Подставим в исходное уравнение и получим
$$v_{xy} + \beta v_x + \alpha v_y + \alpha \beta v + a v_x + a \alpha v + b v_y + \beta b v + cv = \overline{f}(x,y), \overline{f}(x,y) = \frac{f(x,y)}{A}$$
$$v_{xy} + (\beta + a)v_x + (\alpha + b)v_y + (\alpha \beta + a \alpha + b \beta + c)v = \overline{f}$$
Отсюда

\begin{equation*}
	\boxed{
		u = v^{-bx - ay}, \\
		\beta = -a, \alpha = -b,\\
		v_{xy} + (c-ab)v = \overline{f}
	}
\end{equation*}

\end{document}
