\makeatletter
\makeatother
\documentclass[../main.tex]{subfiles}

\graphicspath{
    {img}
	{../img/}
}

\begin{document}
\section{Квазилинейные дифференциальные уравнения с производными 1-го порядка.}


\textit{На экзамене обязательно нужно знать 2 теоремы из предыдущего параграфа (прим. редактора)}


\par Рассмотрим:
\begin{equation}\label{1.4.1}
	\sum_{i=1}^{n}a^{(i)}(x,u)\pderiv{u}{x_i}=f(x,u)
\end{equation}
\\\parТакие уравнения имеют непрерывные производные в окрестности $\Omega(x^{(0)}$ и одновременно не обращяются в ноль в этой точке.\\

\par Решение уравнения (\ref{1.4.1}) ищем в неявном виде:
$$V(x_1,x_2,...,x_n,u)=0$$
$\pderiv{V}{u}$ - существует, гарантирует разрешимость.
\\ \\
$\pderiv{V}{x_i}+\pderiv{V}{u}\pderiv{u}{x_i} = 0$, а $\pderiv{U}{x_i} = \frac{-\pderiv{V}{x_i}}{\pderiv{V}{u}}$ подставим в (\ref{1.4.1}):\\

$$-\sum_{i=1}^{n}a^{(i)}(x,u)\frac{\pderiv{V}{x_i}}{\pderiv{V}{u}}-f(x,u)=0$$
$$\sum_{i=1}^{n}a^{(i)}(x,u)\pderiv{V}{x_i}+f\pderiv{V}{u} = 0$$
Получилось линейное однородное уравнение, зависящее от n независимых переменных.
$$\frac{dx_1}{a^{(1)}}=\frac{dx_2}{a^{(2)}}=...=\frac{dx_n}{a^{(n)}}=\frac{du}{f}$$
Находим n первых интегралов $\psi_1,...,\psi_n$. Формируем $v(\psi_1(x,u),...,\psi_n(x,u))$ и разрешаем относительно u.

\end{document}
