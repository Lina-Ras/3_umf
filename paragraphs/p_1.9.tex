\makeatletter
\makeatother
\documentclass[../main.tex]{subfiles}

\graphicspath{
    {img}
	{../img/}
}

\begin{document}
\section{Приведение к каноническому виду ДУ 2-ого порядка с $n$ независимыми переменными}
Рассмотрим уравнение
\begin{equation}  \label{eq:1.9.1}
    \sum_{i=1}^n\sum_{j=1}^n a_{ij} \frac{\partial^2 u}{\partial x_i \partial x_j} + \sum_{i=1}^n \frac{\partial u}{\partial x_i} + cu = f(x).
\end{equation}

Пусть $C$ -- матрица преобразования, приводящая характеристический многочлен $$\sum_{i=1}^n\sum_{j=1}^n a_{ij} \xi_i \xi_j $$ к каноническому виду 
$$\sum_{i=1}^n a_i \mu_i^2.$$
Введем в (\ref{eq:1.9.1}) замену 
$$x_i = \sum_i^n c_{ji}y_i,$$
где $c_{ji}$ -- элементы $C^T$.
Вычислим производные 
$$\frac{\partial u}{\partial x_i} = \sum^n_{i=k}\frac{\partial u}{\partial y_k} \frac{\partial y_k}{\partial x_i} = \sum_{k=1}^n c_{ik} \frac{\partial u}{\partial y_k}$$
$$\frac{\partial^2 u}{\partial x_i \partial x_j} = \sum^n_{k=1}\sum^n_{s=1} c_{ik} c_{js} \frac{\partial^2 u}{\partial y_k \partial y_s}$$
и, подставив в исходное уравнение, получим
\begin{equation}
\boxed{\sum^n_{k=1}\sum^n_{s=1} A_{ks} \frac{\partial^2 u}{\partial y_k \partial y_s} + \sum_{k=1}^n A_k \frac{\partial u}{\partial y_k} + cu = f}
\end{equation}
$$A_{ks} = \sum^n_{i=1}\sum^n_{j=1} a_{ij} c_{ij} c_{js},$$
$$A_k = \sum_{i=1}^n a_i c_{ik}.$$
Коэффициенты $A$ в приведенном уравнении совпадают с коэффициентами в приведенной квадратичной форме.

Таким образом, чтобы привести ДУ 2-ого порядка с $n$ независимымим переменными к каноническому виду, необходимо:
\begin{enumerate}
    \item Выписать характеристический многочлен исходного уравнения.
    \item Привести характеристический многочлен к каноническому виду.
    \item Выписать матрицу преобразования, приводящего многочлен к каноническому виду.
    \item Протранспонировать матрицу преобразования.
    \item С помощью транспонированной матрицы ввести замену в исходном уравнении.
\end{enumerate}
{\bf В случае $n=3$ замена переменных дает:
\begin{enumerate}
    \item Гиперболический вид
    $$\frac{\partial^2 u }{\partial y_1 ^2} + \frac{\partial^2 u}{\partial y_2^2} - \frac{\partial^2 u}{\partial y_3^2} + a_1 \frac{\partial u}{\partial y_1} + a_2 \frac{\partial u}{\partial y_2} + a_3 \frac{\partial u}{\partial y_3} + cu = f$$ 
    \item Эллиптический вид
    $$\frac{\partial^2 u }{\partial y_1 ^2} + \frac{\partial^2 u}{\partial y_2^2} + \frac{\partial^2 u}{\partial y_3^2} + a_1 \frac{\partial u}{\partial y_1} + a_2 \frac{\partial u}{\partial y_2} + a_3 \frac{\partial u}{\partial y_3} + cu = f$$ 
    \item Параболический вид
    $$\frac{\partial^2 u}{\partial y_2^2} + \frac{\partial^2 u}{\partial y_3^2} + a_1 \frac{\partial u}{\partial y_1} + a_2 \frac{\partial u}{\partial y_2} + a_3 \frac{\partial u}{\partial y_3} + cu = f.$$
\end{enumerate}}

\end{document}