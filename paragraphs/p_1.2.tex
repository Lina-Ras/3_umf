\makeatletter
\makeatother
\documentclass[../main.tex]{subfiles}

\graphicspath{
    {img}
	{../img/}
}

\begin{document}
\section{Классификация ДУ 2-го порядка с 2 независимыми переменными}
Из предыдущего параграфа имеем (\ref{eq:1.1}):
\[
	a_{11}(x,y)\pderiv{u}{x^2} + 2a_{12}(x,y)\pderiv{^2u}{x \partial{y}} +
	a_{22}(x,y)\pderiv{^2u}{y^2} + a(x,y)\pderiv{u}{x} + b(x,y)\pderiv{u}{y} + c(x,y)u = f(x,y)
\]

Введём функцию $D = a_{12}^2-a_{11}a_{22}$ -- дискриминант.
Проведём классификацию по дискриминанту:
\begin{enumerate}
	\item
	      Если в т.$(x,y)$ $D>0$, то исходное уравнение --
	      уравнение гиперболического типа в т.$(x,y)$
	\item
	      Если в т.$(x,y)$ $D=0$, то исходное уравнение --
	      уравнение параболического типа в т.$(x,y)$
	\item
	      Если в т.$(x,y)$ $D<0$, то исходное уравнение --
	      уравнение элиптического типа в т.$(x,y)$
\end{enumerate}

Если значение дискриминанта сохраняется во всей области $\Omega$,
то исходное уравнение называется соответственно уравнением гиперболического,
параболического, элиптического типа во всей области $\Omega$.\\

Бывают случаи, когда область $\Omega$ разбивается:
\[\Omega = \Omega_1 \cup \overline{\Omega_0} \cup \Omega_2, \]
$\Omega_1$ --область, на котором уравнение (\ref{eq:1.1}) является уравнением гиперболического типа.\\
$\overline{\Omega_0}$ --область, на котором уравнение (\ref{eq:1.1}) является уравнением параболического типа.\\
$\Omega_2$ --область, на котором уравнение (\ref{eq:1.1}) является уравнением элиптического типа.\\
В таких случаях исходное уравнение называется уравнением смешанного типа.

\begin{enumerate}
	\item Уравнение гиперболического типа\\
	      Волновое, в частности уравнение колебания струны:
	      \[\pderiv{^2u}{t^2}=a^2\pderiv{^2u}{x^2},\]
	      где $u$ -- смещение колебания струны в момент времени $t$.
	\item Уравнение параболического типа\\
	      Уравнение теплопроводности:
	      \[\pderiv{u}{t}=a^2\pderiv{^2u}{x^2},\]
	      где $u$ -- температура.\\
	      Уравнение описывает изменение температуры в тонком стержне.
	\item Уравнение элиптического типа\\
	      Уравнение Лапласа:
	      \[
		      \pderiv{^2u}{x^2}+\pderiv{^2u}{y^2} = 0,\quad
		      \Delta u=\pderiv{^2u}{x^2}+\pderiv{^2u}{y^2}
	      \]
\end{enumerate}
\end{document}
