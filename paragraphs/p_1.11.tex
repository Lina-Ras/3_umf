\makeatletter
\makeatother
\documentclass[../main.tex]{subfiles}

\graphicspath{
    {img}
	{../img/}
}

\begin{document}
\section{Общее решение простейших ДУ 2-го порядка с 2-мя независимыми переменными}
\subsection{Гиперболическое уравнение.}
Гиперболическое уравнение второго порядка с двумя независимыми переменными может быть приведено к виду:
\[
    \pderivTwo{u}{x}{y} + Cu = f(x, y)
\]
Полагая $C$ = 0, имеем простейшее гиперболическое уравнение:
\begin{equation}
    \pderivTwo{u}{x}{y} = f(x, y)
\end{equation}
Найдем общее решение, интегрируя данное уравнение сперва по $x$, а затем по $y$. \\Заметим, что при интегрировании уравнения с частными производными по одной из независимых переменных возникающие константы в общем случае зависят от остальных переменных.
\[
    \pderiv{u}{y} = \int\limits_{0}^{x}f(\xi, y)\,d\xi + C(y)
\]
\[
    u(x,y) = \int\limits_{0}^{y}\int\limits_{0}^{x}f(\xi, \eta)\,d\xi\,d\eta + \int\limits_{0}^{y}C(\eta)\,d\eta + C_1(x)
\]
Тогда общее решение будет иметь вид:
\begin{equation}
        u(x,y) = \int\limits_{0}^{y}\int\limits_{0}^{x}f(\xi, \eta)\,d\xi\,d\eta + C_1(x) + C_2(y),
\end{equation}
где $C_1(x)$, $C_2(y)$ - произвольные непрерывно дифференцируемые функции.\\
Для \textit{однородного} уравнения
\[
    \pderivTwo{u}{x}{y} = 0
\]
общее решение определяется формулой:
\[
        u(x,y) = C_1(x) + C_2(y)
\]
\subsection{Эллиптическое уравнение}
Так как эллиптическое уравнение второго порядка с двумя независимыми переменными при определенных условиях может быть преобразовано к виду:
\[
    \pderivTwo{u}{x}{x} + \pderivTwo{u}{y}{y} + Cu = f,
\]
то, полагая $C$ = 0, $f$ = 0, получим \textit{уравнение Лапласа}:
\begin{equation}
        \pderivTwo{u}{x}{x} + \pderivTwo{u}{y}{y} = 0,
\end{equation}
Пусть $f(z)$ - произвольная аналитическая функция комплексного переменного $z = x + iy$ в области $\Omega$. Выделим действительную и мнимую части функции $f(z)$, то есть представим её как:
\[
    f(z) = u(x, y) + iv(x,y)
\]
Для любой аналитической функции комплексного переменного
выполнены \textit{условия Коши-Римана}:
\begin{equation}
    \begin{dcases}
        \pderiv{v}{x} + \pderiv{u}{y} = 0,\\
        \pderiv{u}{x} - \pderiv{v}{y} = 0
    \end{dcases}
\end{equation}
Дифференцируя первое уравнение по $x$, второе по $y$, и вычитая из первого второе, имеем:
\[
    \pderivTwo{v}{x}{x} + \pderivTwo{v}{y}{y} = 0
\]
Аналогично, дифференцируя первое по $y$, второе по $x$, и складывая их, имеем:
\[
    \pderivTwo{u}{x}{x} + \pderivTwo{u}{y}{y} = 0
\]
Таким образом, действительная часть $u(x, y)$ и мнимая часть $v(x, y)$ аналитической функции комплексного переменного являются решениями уравнения Лапласа, т. е. простейшего эллиптического уравнения на плоскости.\\
\underline{Важные примеры:}
    \begin{enumerate}
        \item $f(z) = e^z = e^{x+iy} = e^x(\cos{y}+i\sin{y})$.\\ Тогда функции $u(x,y) = e^x\cos{y}$ и $v(x,y) = e^x\sin{y}$ - частные решения уравнения Лапласа на плоскости.
        \item $f(z) = \ln{\dfrac{1}{z-z_0}}$, где $z_0 = x_0 + iy_0 = const$\\
        Запишем комплексное число $z-z_0$ как $z-z_0=re^{i\phi}$, где $r = \sqrt{(x-x_0)^2 + (y - y_0)^2},\\\phi = \arctg\dfrac{y-y_0}{x-x_0}.$\\
        Тогда $f(z) = \ln{\dfrac{1}{r}} - i\arctg\dfrac{y-y_0}{x-x_0}.$\\
        Следовательно, имеем частные решения уравнения Лапласа:
        \[
            u = \ln{\dfrac{1}{\sqrt{(x-x_0)^2 + (y - y_0)^2}}}
        \]
        \[
            v = \arctg\dfrac{y-y_0}{x-x_0}
        \]
        Умножив $u$ на $\dfrac{1}{2\pi}$, получим \textit{фундаментальное решение уравнения Лапласа на плоскости $\R^2$}:
        \begin{equation}
            u(x, y) = G(M, M_0) \equiv \dfrac{1}{2\pi}\ln{\dfrac{1}{R_{MM_0}}} =  \dfrac{1}{2\pi}\ln{\dfrac{1}{\sqrt{(x-x_0)^2 + (y - y_0)^2}}},\; M \ne M_0.
        \end{equation}
        В случае трехмерного пространства $\R^3$ \textit{уравнение Лапласа} имеет вид:
        \begin{equation}
            \pderivTwo{u}{x}{x} + \pderivTwo{u}{y}{y} + \pderivTwo{u}{z}{z} = 0,
        \end{equation}
        Тогда \textit{фундаментальное решением уравнения Лапласа в $\R^3$} будет иметь вид:
        \begin{equation}
            u(x, y, z) = G(M, M_0) \equiv \dfrac{1}{4\pi R_{MM_0}} =  \dfrac{1}{4\pi{\sqrt{(x-x_0)^2 + (y - y_0)^2 + (z - z_0)^2}}},
        \end{equation}
        где $x_0, y_0, z_0$ - координаты фиксированной точки $M_0$.\\
        Убедимся, что $u(x, y, z) = \dfrac{1}{4\pi R_{MM_0}}$ в самом деле является решением уравнения Лапласа в $\R^3$:
        \[
            \pderiv{u}{x} = \frac{\partial}{\partial x}\left(\dfrac{1}{4\pi R_{MM_0}}\right) =  - \dfrac{2(x-x_0)}{{8\pi(\sqrt{(x-x_0)^2 + (y - y_0)^2 + (z - z_0)^2}})^3} = - \dfrac{x-x_0}{4\pi R_{MM_0}^3}
        \]
        Аналогично:
        \[
            \pderiv{u}{y} = - \dfrac{y-y_0}{4\pi R_{MM_0}^3},\;
            \pderiv{u}{z} = - \dfrac{z-z_0}{4\pi R_{MM_0}^3}.
        \]
        Тогда:
        \[
            \pderivTwo{u}{x}{x} = \dfrac{1}{4\pi} \left(-\dfrac{1}{R_{MM_0}^3} + \dfrac{3(x-x_0)^2}{R_{MM_0}^5}\right),
        \]
        \[
            \pderivTwo{u}{y}{y} = \dfrac{1}{4\pi} \left(-\dfrac{1}{R_{MM_0}^3} + \dfrac{3(y-y_0)^2}{R_{MM_0}^5}\right),
        \]
        \[
            \pderivTwo{u}{z}{z} = \dfrac{1}{4\pi} \left(-\dfrac{1}{R_{MM_0}^3} + \dfrac{3(z-z_0)^2}{R_{MM_0}^5}\right).
        \]
        Подставляя эти производные второго порядка в уравнение Лапласа для $\R^3$ получаем тождество:
        \[
            \dfrac{1}{4\pi}\left(-\dfrac{3}{R_{MM_0}^3} + \dfrac{3(x-x_0)^2 + (y-y_0)^2 + (z-z_0)^2)}{R_{MM_0}^5}\right) = 0,
        \]
        \[
            \dfrac{1}{4\pi}\left(-\dfrac{3}{R_{MM_0}^3} + \dfrac{3}{R_{MM_0}^3}\right) = 0,
        \]
        \[
            0 = 0.
        \]
        Следовательно, $u(x, y, z) = \dfrac{1}{4\pi R_{MM_0}}$ в самом деле является фундаментальным решением уравнения Лапласа во всех точках $\R^3$, кроме $M = M_0$.
    \end{enumerate}
\end{document}