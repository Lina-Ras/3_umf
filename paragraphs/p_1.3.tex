\makeatletter
\makeatother
\documentclass[../main.tex]{subfiles}

\graphicspath{
    {img}
	{../img/}
}

\begin{document}
\section{ДУ с частными производными 1-ого порядка}
Решим однородное ДУ 1-ого порядка с $n$ независимыми переменными:
\begin{equation}
	\label{eq:1.3.1}
	\sum_{i=1}^n a^{(i)}(x)\pderiv{u}{x_i} = 0
\end{equation}

$a^{(i)}(x)$ определены и непрерывны вместе с первыми производными в
окрестности точки $x_0^{(0)}$ и не обращаются одновременно в $0$ в этой точке.
\[
	\left((a^{(1)}(x^{(0)}))^2 + (a^{(2)}(x^{(0)}))^2 + \dots + (a^{(n)}(x^{(0)}))^2 \neq 0\right)
\]
Полагаем, что $a^{(n)}(x^{(0)})\neq 0$
В соответсвтии с (\ref*{eq:1.3.1}) запишем систему ДУ:
\begin{equation}
	\label{eq:1.3.2}
	\frac{dx_1}{a^{(1)}} = \frac{dx_2}{a^{(2)}} = \dots = \frac{dx_n}{a^{(n)}}
\end{equation}
Систему (\ref*{eq:1.3.2}) перепишем в виде (n-1) уравнения:
\begin{equation}
	\label{eq:1.3.3}
	dx_1 = \frac{a^{(1)}}{a^{(n)}}dx_n, \:
	dx_2 = \frac{a^{(2)}}{a^{(n)}}dx_n,
	\dots,
	dx_{n-1} = \frac{a^{(n-1)}}{a^{(n)}}dx_n
\end{equation}
Данная система имеет (n-1) независимый интеграл.
\begin{theorem}
	Всякий интеграл системы (\ref*{eq:1.3.3}) является решением системы (\ref*{eq:1.3.1})
\end{theorem}
\begin{proof}
	Пусть $\psi(x)$ -- интеграл системы (\ref*{eq:1.3.3}). Тогда $d\psi \equiv 0$,
	т.е.
	\begin{gather*}
		d\psi = \pderiv{\psi}{x_1}dx_1 + \dots + \pderiv{\psi}{x_n}dx_n = [\ref*{eq:1.3.3}] =
		\pderiv{\psi}{x_1}\frac{a^{(1)}}{a^{(n)}}dx_n + \dots
		+ \pderiv{\psi}{x_{n-1}}\frac{a^{(n-1)}}{a^{(n)}}dx_n + \pderiv{\psi}{x_n}dx_n = \\
		= \left(\sum_{i=1}^{n}a^{(i)}\pderiv{\psi}{x_i}\right)\frac{dx_n}{a^{(n)}} \equiv 0\\
		\sum_{i=1}^{n}a^{(i)}\pderiv{\psi}{x_i} = 0
	\end{gather*}
\end{proof}

\begin{theorem}
	Если $\psi_1, \psi_2, \dots, \psi_{n-1}$ является независимой интегриралами системы (\ref*{eq:1.3.3}),
	то функция $F(\psi_1, \psi_2, \dots, \psi_{n-1})$ -- непрерывная и имеющая непрерывную
	производную по $\psi_i$ -- является решением (\ref*{eq:1.3.1})
\end{theorem}
\begin{proof}
	\begin{gather*}
		\sum_{i=1}^n = a^{(i)}\pderiv{F}{x_i} =
		\sum_{i=1}^n a^{(i)} \sum_{j=1}^{n-1}\pderiv{F}{\psi_j}\pderiv{\psi_j}{x_i} =
		\sum_{j=1}^{n-1} \sum_{i=1}^n a^{(i)} \pderiv{\psi_j}{x_i} = 0
	\end{gather*}
	$F(\psi_1, \psi_2,\dots,\psi_n)$ называетсяя \textit{общим решением} уравнения (\ref*{eq:1.3.1})
\end{proof}
Чтобы найти общее решение уравнения (\ref*{eq:1.3.1}) необходимо,
\begin{enumerate}
	\item Используя полные дифференциалы записать систему (\ref*{eq:1.3.3})
	\item Найти (n-1) независимый интеграл этой системы
	\item Построить функцию от этих интегралов, которая будет являться достаточно гладкой
\end{enumerate}
\end{document}