\makeatletter
\makeatother
\documentclass[../main.tex]{subfiles}

\graphicspath{
    {img}
	{../img/}
}

\begin{document}
\section{Решение задачи Коши для волнового уравнения методом характеристик}
Поставим задачу Коши для однородного уравнения поперечных колебаний струны
\begin{equation}
	\label{eq:2.2.1}
	\frac{\partial^2u}{\partial t^2}-a^2 \cdot \frac{\partial^2u}{\partial x^2}=0, \;
	\D=\lbrace- \infty<x<+\infty, |t|\leq T \rbrace, \quad
	u(x,t) \in C^2(\D)
\end{equation}
\begin{equation}
	\label{eq:2.2.2}
	u|_{t=0} = \varphi(x), -\infty<x<+\infty, \quad \varphi \in C^2(\R^1)
\end{equation}
\begin{equation}
	\label{eq:2.2.3}
	\pderiv{u}{t}|_{t=0} = \psi(x),  -\infty<x<+\infty, \quad \psi \in C^1(\R^1)
\end{equation}
где $a=\sqrt{\frac{N}{p}}$; \quad N - натяжение струны; \quad p - линейная плотность струны;\\
t - временная переменная; \quad x - пространственная переменная.\\
(\ref{eq:2.2.1}) - поперечные колебания тонкой упругой струны.\\
u(x, y) - график струны в момент времени t. Отклонение точки струны с координатой x от оси Ox\\
(\ref{eq:2.2.2}) - начальный график струны.\\
(\ref{eq:2.2.3}) - скорость в начальный момент времени.\\
\begin{center}
	{\largeМетод Характеристик}\\
	\begin{enumerate}
		\item
		      Уравнение приводится к каноническому виду
		\item
		      Находим общее решение
		\item
		      Путем подставления в начальное условие определяется единственное решение.
	\end{enumerate}
\end{center}

\subsection{Формула Даламбера}
\begin{equation*}
	(dx)^2-a^2(dt)^2=0
\end{equation*}
Для гиперболического уравнения на основании характеристических уравнений (\ref{eq:2.2.2}) и (\ref{eq:2.2.3}),
найдем два семейства характеристик на плоскости $Oxt$:
\[x-at=C_1, \qquad  x+at=C_2\]

Выполним замену переменных:\\
$
	\begin{left}
		\{
		\begin{array}{lcl}
			\xi=x-at  \\
			\eta=x+at \\
		\end{array}
		\right.
	\end{left}
$\\

\begin{gather*}
	\text{Приведём гиперболическое уравнение к каноническому виду:}\\
	\pderiv{u}{t}=a\pderiv{u}{\xi}+a\pderiv{u}{\eta}\\
	\pderivt{u}{t}=a^2\pderivt{u}{\xi}+a^2\pderivt{u}{\eta}-2a^2\frac{\partial^2 u}{\partial\xi\partial\eta}\\
	\pderiv{u}{x}=\pderiv{u}{\xi}+\pderiv{u}{\eta}\\
	\pderivt{u}{x}=\pderivt{u}{\xi}+\pderivt{u}{\eta}+2\frac{\partial^2 u}{\partial\xi\partial\eta} \\
	-4a^2\frac{\partial^2 u}{\partial\xi\partial\eta}=0 \\
	\frac{\partial^2 u}{\partial\xi\partial\eta}=0\\
	u(\xi,\eta)=C_1(\xi)+C_2(\eta)\\
\end{gather*}

Получаем общее решение однородного уравнения колебаний струны:\\
\[u(x,t)=C_1(x-at)+C_2(x+at)\]\\
Определим неизвестные функции $C_1(x), C_2(x)$ из начальных условий.\\\\
$
	\begin{left}
		\{
		\begin{array}{lcl}
			C_{1}(x)+C_2(x)=\varphi(x) \\
			-aC_1'(x)+aC_2'(x)=\psi(x) \\
		\end{array}
		\right.
	\end{left}
$
, где $C_{1,2}'(x)$- производные по переменной x.\\
\\Интегрируем второе уравнеие:\\\\
$
	\begin{left}
		\{
		\begin{array}{lcl}
			-C_1(x)+C_2(x)=\frac{1}{a}\int_{x_0}^x\psi(\xi)d\xi+C \\
			C_1(x)+C_2(x)=\varphi(x)                              \\
		\end{array}
		\right.
	\end{left}
$
\\\\
$C_2(x)=\frac{1}{2a}\int_{x_0}^x\psi(\xi)d\xi+\frac{C}{2}+\frac{\varphi(x)}{2}$\\\\
$C_1(x)=\frac{\varphi(x)}{2}-\frac{1}{2a}\int_{x_0}^x\psi(\xi)d\xi-\frac{C}{2}$\\\\
Подставляем полученные значения в общее решение:\\\\
$u(x,t)=\frac{\varphi(x-at)}{2}-\frac{1}{2a}\int_{x_0}^{x-at}\psi(\xi)d\xi-\frac{C}{2}+\frac{1}{2a}\int_{x_0}^{x+at}\psi(\xi)d\xi+\frac{C}{2}+\frac{\varphi(x+at)}{2}$
\begin{equation}\label{eq:2.2.4}
	u(x,t)=\frac{\varphi(x-at)+\varphi(x+at)}{2}+\frac{1}{2a}\int_{x-at}^{x+at}\psi(\xi)d\xi
\end{equation}
(\ref{eq:2.2.4}) - формула Даламбера или формула решения задачи Коши для волнового уравнения.
\end{document}
