\makeatletter
\makeatother
\documentclass[../main.tex]{subfiles}

\graphicspath{
    {img}
	{../img/}
}

\begin{document}
\section{Общее понятие о дифференциальных уравнениях с частными производными}
Рассмортим пространство $\R^n, x\in\R^n, x=(x_1, x_2, \dots, x_n)$\\
Выделим область $\Omega \subset \R^n$ и рассмторим функцию $u$, определённую на области
$\Omega$\\
\textit{Первой производной по переменной $x_i$ в фиксированной точке $x$} называют
\[
    \pderiv{u}{x_i} = u_{x_i} =
    \lim_{\Delta x_i \to 0}
    \frac{u(x_1, x_2, \dots, x_i+\Delta x_i, \dots, x_n) - u(x_1,\dots, x_n)}
        {\Delta x_i}
\]
Если предел существует в каждой точке области $\Omega$, то функцию $u$ называют \textit{диффиренцируемой функцией}
 в области $\Omega$, а функция $\pderiv{u}{x_i}$ определена во всей области $\Omega$

\end{document}