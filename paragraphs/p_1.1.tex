\makeatletter
\makeatother
\documentclass[../main.tex]{subfiles}

\graphicspath{
    {img}
	{../img/}
}

\begin{document}
\section{Общее понятие о дифференциальных уравнениях с частными производными}
Рассмотрим пространство $\R^n$ и точку  $x\in\R^n, x=(x_1, x_2, \dots, x_n)$\\
Выделим область $\Omega \subset \R^n$ и рассмотрим функцию $u(x)=u(x_1, x_2, \dots, x_n)$, определённую на области
$\Omega$.

\begin{definition}
	Первой производной по переменной $x_i$ в фиксированной точке $x$ называют
	\[
		\pderiv{u}{x_i} = u_{x_i} =
		\lim_{\Delta x_i \to 0}
		\frac{u(x_1, x_2, \dots, x_i+\Delta x_i, \dots, x_n) - u(x_1,\dots, x_n)}
		{\Delta x_i}
	\]
\end{definition}

Если предел существует в каждой точке области $\Omega$, то функцию $u(x)$ называют \textit{дифференцируемой функцией}
в области $\Omega$, а функция $\pderiv{u}{x_i}$ определена во всей области $\Omega$\\
Производные более высоких порядков определяются по индукции.
\[
	\pderiv{^2u(x)}{x_i\partial{x_j}} = \pderiv{g(x)}{x_j}, \text{где } g(x)=\pderiv{u(x)}{x_i}
\]
\begin{definition}
	Множество $C^m(\Omega)$ называется пространством $m$ раз непрерывно дифференцируемых функций в области $\Omega$.
	$u(x) \in C^m(\Omega)$, если определены и непрерывны в области $\Omega$ все
	частные производные до порядка $m$ включительно.
\end{definition}
Рассмотрим функцию $F(x_1, x_2, \dots, x_n,z_1,z_2,\dots, z_N) \in C(\R^{n+N})$, имеющую
непрерывные производные по $x_i, i=\overline{1,n}$

\begin{definition}
	Отношение
	\[F\left(x,u, \pderiv{u}{x_1}, \pderiv{u}{x_2},\dots, \pderiv{u}{x_n}, \pderiv{^2u}{x_1^2}, \dots, \pderiv{^mu}{x_n^m}\right) = 0\]
	называетя дифференциальным уравнением с частными производными.
\end{definition}

\textit{Порядком уравнения} называется порядок старшей производной,
входящей в дифференициальное уравнение.
Если ввести оператор $Lu=F, L:C^m(\Omega) \rightarrow C(\Omega),$ то дифференциальное уравнение
с частными производными можно записать в виде $Lu=f$.\\

Уравнение называется \textit{линейным}, если выполнятеся:
\[L(\al u)=\al Lu\]
\[L(u_1+u_2)=Lu_1 + Lu_2,\]
\[u_1, u_2 \in C^m(\Omega)\]

Введём мультииндекс $\al = (\al_1,\al_2,\dots,\al_n)$ и приведём дифференциальные операторы
\[D_i=\pderiv{ }{x_i},D^{\al_i}_i=\pderiv{^{\al_i }}{x_i^{\al_i}}, D^\al = D^{\al_1}_1 D{^\al_2}_2\dots D^{\al_n}_n\]
Линейное дифференициальное уравнение порядка $m$ можно записать в виде:
\[
	\sum_{0 \leq \al_1+\al_2+\dots+\al_n \leq m}
	a_{\al(\al_1, \al_2, \dots, \al_n)}
	\pderiv{^{\al_1+\al_2+\dots+\al_n}u}
	{x_1^{\al_1} \partial{x_2^{\al_2}}, \dots, \partial{x_n^{\al_n}}}
\]
Учитывая введённые операторы:
\[
	\sum_{0 \leq |\al| \leq m}a_\al D^\al u = 0
\]

\begin{definition}
	Часть уравнения, содержащая старшие производные называется
	главной частью уравнения.
	\[L_0u = \sum_{|\al|=m} a_\al D^\al u\]
\end{definition}
Линейное дифференциальное уравнение порядка 2 можно записать как:
\[
	\sum^n_{i=1}\sum^n_{j=1}a_{ij}\pderiv{^2u}{x_i \partial{x_j}} + 
	\sum^n_{i=1}a_i\pderiv{u}{x_i} + eu = f(x)
\]
Линейное дифференциальное уравнение порядка 2 с двумя 
независимыми переменными можно записать как:
\begin{equation}
	\label{eq:1.1} %first chapter first paragraph first ref
	a_{11}(x,y)\pderiv{u}{x^2} + 2a_{12}(x,y)\pderiv{^2u}{x_i \partial{x_j}} +
	a_{22}(x,y)\pderiv{u}{y^2} + a(x,y)\pderiv{u}{x} + b(x,y)\pderiv{u}{y} + c(x,y)u = f(x,y)
\end{equation}
Введём в соответсвтие производной по $x$ координату $\xi_1=\pderiv{ }{x}$,
для $y$ -- $\xi_2 = \pderiv{ }{y}$. Определим вектор $\xi=(\xi_1, \xi_2)$.\\
Тогда для главной части уравнения можно записать характеристический многочлен
\[
	p(\xi)=a_{11}\xi_1^2 + 2a_{12}\xi_1\xi_2 + a_{22}\xi_2^2
\]
\end{document}
