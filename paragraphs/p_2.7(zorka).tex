\makeatletter
\makeatother
\documentclass[../main.tex]{subfiles}

\graphicspath{
    {img}
	{../img/}
}

\begin{document}
\section{Решение задачи Коши для волнового уравнения}
\begin{enumerate}
	\item $n=2$
	      \begin{equation}
		      \label{eq:2.7.1_z}
		      \begin{split}
			      \pderiv{^2u}{t^2} = a^2\left(\pderiv{^2u}{x^2}+\pderiv{^2u}{y^2}\right)\\
			      u|_{t=0} = \varphi(x,y)\\
			      \pderiv{u}{t}|_{t=0} = \psi(x,y)
		      \end{split}
	      \end{equation}
	      Решение записывается по формуле Кирхгофа.
	      \begin{gather*}
		      u(x,y,t) = \frac{1}{2\pi a}\pderiv{ }{t}\left(
		      \int\displaylimits_{|\xi-x|<at} \frac{\varphi(\xi)d\xi}{\sqrt{a^2t^2 - |\xi-x|^2}}\right)
		      +
		      \frac{1}{2\pi a}\left(
		      \int\displaylimits_{|\xi-x|<at} \frac{\psi(\xi)d\xi}{\sqrt{a^2t^2 - |\xi-x|^2}}\right)\\
		      x \rightarrow \xi_1\\
		      y \rightarrow \xi_2\\
		      \xi = (\xi_1,\xi_2)
	      \end{gather*}
	      Интегрируем по кругу с центром $M(x,y)$ радиуса $|\xi-t|$. Для этого перейдём к
	      полярным координатам:
	      \begin{gather*}
		      \xi_1 = x + r cos(\varphi), \qquad \xi_2 = y + r sin(\varphi)\\
		      d\xi = r dr d\varphi \\
		      0 \leq r \leq at, \qquad 0 \leq \varphi \leq 2\pi
	      \end{gather*}

	\item $n=3$
	      \begin{equation}
		      \label{eq:2.7.2_z}
		      \begin{split}
			      \pderiv{^2u}{t^2} = a^2\left(\pderiv{^2u}{x^2}+\pderiv{^2u}{y^2}+\pderiv{^2u}{z^2}\right)\\
			      u|_{t=0} = \varphi(x,y,z)\\
			      \pderiv{u}{t}|_{t=0} = \psi(x,y,z)
		      \end{split}
	      \end{equation}
	      Решение записывается по формуле Пуассона.
	      \begin{gather*}
		      u(x,y,z,t) = \frac{1}{4\pi a^2}\pderiv{ }{t}\left(
		      \frac{1}{t}\int\displaylimits_{|\xi-x|=at} \varphi(\xi)d\xi\right)
		      +
		      \frac{1}{4\pi a^2}\left(
		      \frac{1}{t}\int\displaylimits_{|\xi-x|=at} \psi(\xi)d\xi\right)\\
		      x \rightarrow \xi_1\\
		      y \rightarrow \xi_2\\
		      z \rightarrow \xi_3\\
		      \xi = (\xi_1,\xi_2, \xi_3)
	      \end{gather*}
	      Интегрируем по сфере радиуса $at$. Для этого перейдём к
	      полярным координатам:
	      \begin{gather*}
		      \xi_1 = x + r cos(\varphi) sin(\varTheta), \\
		      \xi_2 = y + r sin(\varphi) sin(\varTheta),\\
		      \xi_3 = z + r cos(\varTheta),\\
		      d\xi = r^2 sin(\varTheta) d\varphi d\varTheta \\
		      0 \leq \varphi \leq 2\pi, \qquad 0 \leq \varTheta \leq \pi
	      \end{gather*}
\end{enumerate}
\end{document}
