+\makeatletter
\makeatother
\documentclass[../main.tex]{subfiles}

\graphicspath{
    {img}
	{../img/}
}
\begin{document}
\section{Приведение к каноническому виду ДУ 2-го порядка с двумя независимыми переменными}
Запишем уравнение:
\begin{equation}\label{eq:1.7.1}
	a_{11}\frac{\partial^2u}{\partial x^2} + 2a_{12}\frac{\partial^2u}{\partial x \partial y} + a_{22}\frac{\partial^2u}{\partial y^2} + a_1\pderiv{u}{x} +a_2\pderiv{u}{y}+a_3 = f(x,u) \end{equation} \\
\begin{equation}
	a_{11}(dy)^2 - 2a_{12}dydx + a_22(dx)^2 = 0 \end{equation} \\
$1)\;D = a_{12}^2 - a_{11}a_{22} > 0$
$$\begin{dcases}\phi(x,y) = С_1 \\ \psi(x,y) = С_2\end{dcases}$$ удовлетворяют уравнению:
\\ \\
\begin{equation}\label{eq:1.7.2}
	a_{11}(\pderiv{z}{x})^2 + 2a_{12}\pderiv{z}{x}\pderiv{z}{y} + a_{22}(\pderiv{z}{y})^2 = 0
\end{equation}
\\
Замена:
$$\begin{dcases}\xi = \phi(x,y) \\ \eta = \psi(x,y) \end{dcases}$$
$\overline{a_{12}} = 0, \overline{a_{22}}=0$ \\
$\overline{a_{12}}\frac{\partial^2u}{\partial\xi\partial\eta}=F(\xi, \eta, \pderiv{u}{\xi}, \pderiv{u}{\eta})$ \\
$\overline{a_{12}}\neq0$ (при невырожденном преобр. порядок и тип ур-я сохраняются) \\
$\frac{\partial^2u}{\partial\xi\partial\eta}=f(\xi, \eta, \pderiv{u}{\xi}, \pderiv{u}{\eta})$ \\
\par Если ур-я $\phi(x,y) = C_1$ и $\psi(x,y)=C_2$ разрешимы относительно $y$, то $y=f_1(x, C_1), y=f_2(x, C_2)$ - два семейства линий, которые являются характеристическими линиями для уравнения гиперболического типа. \\

$2)\;D = 0.\; \phi(x, y) = C_1$
$$\begin{dcases}\xi = \phi(x,y) \\ \eta = \psi(x,y) \end{dcases}$$
, где $\psi$ - достаточно гладкая функция и замены удовлетворяют:
$$\begin{vmatrix}
		\phi_x \  & \ \phi_y
		\\
		\psi_x \  & \ \psi_y
	\end{vmatrix} \ne 0
$$ \\
$\phi$ удовлетворяет (\ref{eq:1.7.2}) $\implies \overline{a_{11}}=0.$ \\
\begin{equation}
	a_{11}(\phi_x)^2 + 2a_{12}\phi_x\phi_y + a_{22}(\phi_y)^2=0 \end{equation} \\
\begin{equation}\label{eq:1.7.3}
	a_{12}=\sqrt{a_{11}a_{22}} \implies (\sqrt{a_{11}}\phi_x + \sqrt{a_{22}}\phi_y)^2=0 \end{equation} \\
$\overline{a_{12}}=a_{11}\phi_x\psi_x + a_{12}(\phi_x\psi_y + \psi_x\phi_y)+a_{22}\phi_y\psi_y = \sqrt{a_{11}}\psi_x + \sqrt{a_{22}}\psi_y) + \sqrt{a_{22}}\phi_y(\sqrt{a_{11}}\psi_x + \sqrt{a_{22}}\psi_y) = (\sqrt{a_{11}}\phi_x + \sqrt{a_{22}}\phi_y)(\sqrt{a_{11}}\psi_x + \sqrt{a_{22}}\psi_y) = 0$ \\
\\
$\frac{\partial^2u}{\partial\eta^2} = F(\xi, \eta, \pderiv{u}{\xi}, \pderiv{u}{\eta})$\\
Если $\phi(x,y)=C_1 \implies y = f_1(x, C_1)$ - одно семейство линий, называаемых характеристическими линиями для уравнения параболического типа.\\
$3)\;D<0 \;\; \Phi$ - комплекснозначный 1-й интеграл. \\
$\phi = Re\Phi, \; \psi = Im\Phi$\\
$z = \phi + i\psi$ - удовлетворяет (\ref{eq:1.7.2}) $\implies$\\
$$\begin{dcases} \xi = \phi \\ \eta = \psi \end{dcases}$$\\
\begin{equation}
	a_{11}(\phi_x+i\psi_x)^2 + 2a_{12}(\phi_x + i\psi_x)(\phi_y + i\psi_y) + a_{22}(\phi_y + i\psi_y)^2 = 0 \end{equation} \\
$a_{11}(\phi_x)^2 + 2a_{12}\phi_x\phi_y + a_{22}(\phi_y)^2 - (a_{11}(\psi_x)^2 + 2a_{12}\psi_x\psi_y + a_{22}(\psi_y)^2) + 2i(a_{11}\phi_x\psi_x + a_{12}(\phi_x\psi_y + \phi_y\psi_x)+a_{22}\phi_y\psi_y)=0$ \\
$\xi = \phi(x,y),\; \eta = \psi(x,y)$ \\
$\overline{a_{11}} - \overline{a{22}} + 2i\overline{a{12}} = 0$ \\
$\overline{a_{11}}=\overline{a_{22}}, \; \overline{a_{12}}=0$ \\
$\frac{\partial^2u}{\partial{\xi^2}} + \frac{\partial^2u}{\partial{\eta^2}} = F(\xi, \eta, u, \pderiv{u}{\xi}, \pderiv{u}{\eta})$ \\
Не имеет характеристических линий.
\par Замечание 1. Т. к. производящая функция от 1-го интеграла - это 1-й интеграл, то приведение к каноническому виду неоднозначно.
\par Замечание 2. Существование 1-ых интегралов обеспечивается т. Ковалевской, но она доказывает существование локального, то есть в некоторой окрестности от точки, решения, а не решения на всей области. Следовательно, привести ДУ к каноническому виду можно в окрестности каждой точки, а не на всей области.
\end{document}
