\makeatletter
\makeatother
\documentclass[../main.tex]{subfiles}

\graphicspath{
    {img}
	{../img/}
}
\begin{document}
\section{Классификация ДУ 2-го порядка с n независимыми переменными}

\par Рассмотрим в $\Omega \subset \R^n $ линейное дифференциальное уравнение 2-го порядка с n независимыми переменными:
\begin{equation}\label{eq:1.8.1}
	\sum_{i=1}^n\sum_{j=1}^n a_{ij}(x)\frac{\partial^2u}{\partial x_i \partial x_j} + \sum_{i=1}^n a_i(x) \pderiv{u}{x_i} + c(x)u = f(x)
 \end{equation}

$a_{ij} \in C^2(\Omega), \; a_{ij} = a_{ji} $
\\
Привести к каноническому виду можно только уравнение с постоянными коэффициентами.

Главная часть уравнения:
$$L_0u = \sum_{i=1}^n \sum_{j=1}^n a_{ij} \frac{\partial^2u}{\partial x_i \partial x_j}$$
Характеристический многочлен: \\
\begin{equation}\label{eq:1.8.2}
	P(\xi) = \sum_{i=1}^n \sum_{j=1}^n a_{ij}\xi_i\xi_j,
\end{equation}
где $\frac{\partial}{\partial x_i} \rightarrow \xi_i$\\
\\
(\ref*{eq:1.8.2}) - квадратичная форма, для которой существует невырожденное преобразование, приводящее её к каноническому виду\\
\\
Рассмортим преобразование:
$\xi_i = \sum_{j=1}^n c_{ij} \mu_j $,
где $c_{ij}$ - элементы матрицы, приводящей (\ref{eq:1.8.2}) к каноническому виду. \\
$$\xi_i = \sum_{k=1}^n c_{ik}\mu_k , \; \xi_j = \sum_{s=1}^n c_{js}\mu_s$$
Подставим в (\ref{eq:1.8.2}): \\
$$P(\xi) = \sum_{i=1}^n \sum_{j=1}^n a_{ij} \sum_{k=1}^n c_{ik} \sum_{s=1}^n c_{js}\mu_s\mu_k = \sum_{k=1}^n\sum_{s=1}^n A_{ks}\mu_k\mu_s,$$
где $A_{ks} = \sum_{i=1}^n\sum_{j=1}^n a_{ij}c_{ik}c_{js} $ \\
Существует невырожденное преобразование, которое приведёт к виду:
\begin{equation}
	\label{eq:1.8.3}
	\sum_{i=1}^n a_i\mu_i^2, \quad a_i = 0,-1,1
\end{equation}
\begin{gather*}
	A_{ks}=0, k \neq s; \\
	A_{ii} = a_i\\
\end{gather*}
Количество 0,-1,1 не зависит от выбора преобразования. \\
Классификация (\ref{eq:1.8.1}) проводится по каноническому виду формулы (\ref{eq:1.8.3}): \\
\par Если в каноническом виде все $a_i = 1$ или $-1, \; i=\overline{1,n}$, то искомое уравнение \textbf{эллиптического типа}. \\
\par Если $a_1 = 1, a_i = -1, \; i = \overline{2,n} $ или $a_1 = -1, a_i = 1, \; i = \overline{2,n}$ , то искомое уравнение - \textbf{гиперболического типа}. \\
\par Если $a_1 = 0, a_i = 1, \; i = \overline{2,n} $ или $a_1=0, a_i = -1, \; i = \overline{2,n}$, то исходное уравнение - \textbf{параболического типа}. \\
\end{document}
