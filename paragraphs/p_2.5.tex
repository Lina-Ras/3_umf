\makeatletter
\makeatother
\documentclass[../main.tex]{subfiles}

\graphicspath{
    {img}
	{../img/}
}

\begin{document}
\section{Физическая и геометрическая интерпретация формулы Даламбера}
\begin{equation}
	\label{eq:2.5.1}
	\pderiv{^2u}{t^2}=a^2\pderiv{^2u}{x^2}
\end{equation}
\begin{equation}
	u|_{t=0} = \phi(x)
\end{equation}
\begin{equation}
	\pderiv{u}{t}|_{t=0} = \psi(x)
\end{equation}
\begin{equation}
	u(x, t) = \frac{\phi(x + at) + \phi(x - at)}{2} + \frac{1}{2a}\int_{x-at}^{x+at}\psi(\xi)d\xi
\end{equation}
Общее решение \eqref{eq:2.5.1}:
$$
	u = C_1(x-at) + C_2(x+at)
$$
Пусть $C_2 \equiv 0$, рассмотрим:
$$
	u(x, t) = C_1(x - at)
$$
Пусть наблюдатель находится в точке (в начальный момент времени $t = 0$) с коордиантой $x = C$. Начинаем равномерное движение в положительном направлении оси $x$ со скоростью $a$.
$ x = C + at$ -- координаты наблюдателя в момент времени $t$. Следовательно, для этого наблюдения $x - at = C$ -- постоянно. Т.е. профиль струны остается постоянным ($u = C_1(C)$).
Такое явление называется распространением прямой волны. Функция $C_2(x + at)$ -- распространение обратной волны.

Метод для нахождения профиля струны в произвольный момент времени:
\begin{enumerate}
	\item Строим кривые $C_1(x)$ и $C_2(x)$, описывающие прямую и обратную волну в начальный момент времени.
	\item Раздвигаем эти прямые в противоположных направлениях с постоянными скоростями.
	\item Профиль струны в начальный момент времени получается как алгебраическая сумма ординат развдинутых кривых.
\end{enumerate}
Рассмотрим начальный профиль волны в виде равнобедренного треугольника($\Delta t = \frac{x_2-x_1}{8a}$):

\includegraphics[scale=0.5]{graph_1.jpg}

Введём точку $M(x_0, t_0)$ и посмотрим от значений в каких точках функций
$\phi$ и $\psi$ зависит решение $u(x_0, t_0) = \frac{\phi(x_0-at_0) + \phi(x_0+at_0)}{2} + \frac{1}{2a}\int_{x_0-at_0}^{x_0+at_0}\psi(\xi)d\xi$.

\includegraphics[scale=0.5]{2.5_2.png}

Через $M$ проведём две характеристики:
$$x-at=C_1 \quad x+at=C_2$$ 
Т.к. характеристики проходят через $М(x_0, t_0)$, то получим, что $C_1 = x_0 - at_0$, $C_2 = x_0 + at_0$.
Характеристики пересекают ось Ox в точках $(x_1,0)$ и $(x_2,0)$, т.о. $x_1 = C_1$, $x_2 = C_2$.
Получим, что $x_1 = x_0 -at_0$, $x_2 = x_0 + at_0$,
а следовательно $u(t, x) = \frac{\phi(x_1) + \phi(x_2)}{2} + \int_{x_1}^{x_2}\psi(\xi)d\xi$,
Т.е. решение в произвольной точке $M(x_0, t_0)$ зависит от значений функции $\phi$
в точках $x_1$ и $x_2$ и от значений $\psi$ на отрезке $[x_1, x_2]$.

Треугольник $MPQ$ называется \textit{характеристическим треугольником} и полностью определеяет решение в точке $M$.
Если изменить $\phi$ и $\psi$ вне треугольника, то решение не изменится.
\end{document}
