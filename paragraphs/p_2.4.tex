\makeatletter
\makeatother
\documentclass[../main.tex]{subfiles}

\graphicspath{
    {img}
	{../img/}
}

\begin{document}
\section{Пример некорректно поставленой задачи по Адамару}
\begin{large}
	\begin{center}
		$\pderivt{u}{t}-a^2\pderivt{u}{x}=0$, $\D=\{ -\infty<x<+\infty, |y|\leq T\rbrace$\\
		$ $\\
		$ u|_{y=0} = \varphi(x)$, $\pderiv{u}{y}|_{y=0} = \psi(x)$, $ -\infty<x<+\infty$\\
		$ $\\
	\end{center}
\end{large}

Если говорить о существовании и единственности, то исходное уравнение является уравнением типа Ковалевской и разрешимо относительно старшей производной, а следовательно в классе функций существует и единственно локальное решение этой задачи:\\

Рассмотрим 2 задачи:\\

\begin{center}

	\begin{tabular}{c c}

		$
			\begin{dcases}
				\pderivt{u_1}{x}+\pderivt{u_1}{y}=0, \\\\
				u_1|_{y=0}=0,                        \\\\
				\pderiv{u_1}{y}|_{y=0}=0,
			\end{dcases} $
		 &
		$\begin{dcases}
				 \pderivt{u_2}{x}+\pderivt{u_2}{y}=0, \\\\
				 u_2|_{y=0}=0,                        \\\\
				 \pderiv{u_2}{y}|_{y=0}=e^{-\sqrt{n}}cos(nx),
			 \end{dcases} $
	\end{tabular}

\end{center}
где n - фиксированный положительный параметр.

Решения данных задач определяются выражениями

\begin{center}
	\begin{tabular}{c c c}
		$u_1=0$, &  & $u_2=\frac{1}{n}e^{-\sqrt{n}}cos(nx)sh(ny).$
	\end{tabular}
\end{center}

Введем пространства функций $V_1=V_2=C_0^A(R^1), V=C_0^A(D)$, где $C_0^A$ - пространство ограниченных аналитических функций с соответствующими метрическими расстояниями:
\begin{center}
	$p_1(\varphi_1,\varphi_2)=p_2(\varphi_1, \varphi_2) =||\varphi_1-\varphi_2||_C = \sup_{-\infty<x<+\infty}|\varphi_1(x)-\varphi_2(x)|, $\\
	$ $\\
	$p(u_1, u_2)=||u_1-u_2||_C=\sup_{(x,y) \in D}|u_1(x,y)-u_2(x,y)|.$\\
	$ $\\
	Если выполняется $p(\varphi_1,\varphi_2)=|\varphi_1-\varphi_2|=0<\delta$\\
	$ $\\
	и $p(\psi_1,\psi_2)=|\psi_1-\psi_2|=e^{-\sqrt(n)}<\delta$, при $n>(\ln{\delta})^2$
\end{center}

тогда должно выполняться равенство

\begin{center}
	$p(u_1,U_2)=|u_1-u_2|=\frac{1}{n}e^{-\sqrt{n}}sh(nT)<\epsilon$.
\end{center}

Очевидно, что последнее неравенство не выполняется при достаточно больших n, так как \[\lim_{n \to \infty}{\frac{1}{n}}e^{-\sqrt{n}}sh(nT)=\infty\]\\

Таким образом, задача Коши для эллиптического уравнения поставлена некорректно, так как не выполнено условие корректности.
\end{document}
