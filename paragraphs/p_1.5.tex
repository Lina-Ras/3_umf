\makeatletter
\makeatother
\documentclass[../main.tex]{subfiles}

\graphicspath{
    {img}
	{../img/}
}
\begin{document}
\section{Системы дифференциальных уравнений с частными производными.}
$$U_1(x_1,...,x_n),U_2(x_1,...,x_n),...U_k(x_1,...,x_n).$$ \\
Рассмотрим систему из k уравнений: 

$$\begin{dcases}
F_1(x_1, x_2,..., x_n, z_1, z_2,..., z_N_1), \\
F_2(x_1, x_2,..., x_n, z_1, z_2,..., z_N_2), \\
......................................... \\
F_k(x_1, x_2,..., x_n, z_1, z_2,..., z_N_k)
\end{dcases} $$ \\
Имеем:
$$\begin{dcases}
F_1(x_1, x_2,..., x_n, \pderiv{u_1}{x_1},...,\pderiv{u_k}{x_n},..., \frac{\partial ^{m_1} u_k}{\partial x^{m_1}_n}) = 0, \\
F_2(x_1, x_2,..., x_n, \pderiv{u_1}{x_1},...,\pderiv{u_k}{x_n},..., \frac{\partial ^{m_2} u_k}{\partial x^{m_2}_n}) = 0, \\
........................................................ \\
F_1(x_1, x_2,..., x_n, \pderiv{u_1}{x_1},...,\pderiv{u_k}{x_n},..., \frac{\partial ^{m_k} u_k}{\partial x^{m_k}_n}) = 0, \\
\end{dcases} $$ \\
Рассмотрим линейную систему 1-го порядка:\\
$$
\begin{dcases}
b_{11}\pderiv{u}{x} + b_{12}\pderiv{u}{y} + C_{11}\pderiv{v}{x} + C_{12}\pderiv{v}{y}+b_1u + C_1v = f_1 \\ 
b_{21}\pderiv{u}{x} + b_{22}\pderiv{u}{y} + C_{21}\pderiv{v}{x} + C_{22}\pderiv{v}{y}+b_2u + C_2v = f_2
\end{dcases}$$\\
Представим в матричном виде:\\
$$Lu = f$$ \\
Где:
$$
L = \begin{bmatrix}
C_{11}\pderiv{}{x} + C_{12}\pderiv{}{y} + C_1 \ & \ b_{11}\pderiv{}{x} + b_{12}\pderiv{}{y} + b_1 \\ \\ 
C_{21}\pderiv{}{x} + C_{22}\pderiv{}{y} + C_2 \ & \ b_{21}\pderiv{}{x} + b_{22}\pderiv{}{y} + b_2
\end{bmatrix}
$$ \\ 
$$u = \begin{bmatrix}
u \\ v
\end{bmatrix}, \  f = \begin{bmatrix}
f_1 \\ f_2
\end{bmatrix}$$ \\
Главная часть: \\ 
$$
L_0 = \begin{bmatrix}
C_{11}\pderiv{}{x} + C_{12}\pderiv{}{y} \ & \ b_{11}\pderiv{}{x} + b_{12}\pderiv{}{y} \\ \\ 
C_{21}\pderiv{}{x} + C_{22}\pderiv{}{y} \ & \ b_{21}\pderiv{}{x} + b_{22}\pderiv{}{y} 
\end{bmatrix}
$$ \\ \\
Умножим $L_0u$, а также проведем замену: $\pderiv{}{x} \rightarrow \xi_1, \pderiv{}{y} \rightarrow \xi_2$. \\
\\ Запишем характеристическую матрицу:
$$
A = \begin{bmatrix}
C_{11}\xi_1 + C_{12}\xi_2 \ & \ b_{11}\xi_1 + b_{12}\xi_2 \\ \\ 
C_{21}\xi_1 + C_{22}\xi_2 \ & \ b_{21}\xi_1 + b_{22}\xi_2 
\end{bmatrix}
$$ \\ \\
Характеристический многочлен:\\
$$P(\xi) = detA = a_{11}\xi_1^2 + 2a_{12}\xi_1\xi_2+a_{22}\xi_2^2$$

$$a_{11} = C_{11}b_{21} - C_{21}b_{11}, \ \ 
a_{22} = C_{12}b_{22} - C_{22}b_{12}$$
$$a_{12} = \frac{1}{2}(C_{11}b_{22} + C_{12}b_{21} - C_{21}b_{12} - C_{22}b_{11})$$ \\ 
Классификация исходной системы проводится таким же образом, как и классификация дифференциального уравнения 2-го порядка:
\\$$D = a_{12}^2 - a_{11}a_{22} $$ \\
$D > 0$ - гиперболический тип \\
$D = 0$ - параболический тип \\
$D < 0$ - эллиптический тип \\
\\
\\
\textbf{Пример:}\\
Определить тип системы
$$
\begin{dcases}
2\pderiv{u}{x} + 3\pderiv{u}{y}-3\pderiv{v}{y}+u=0 \\
-\pderiv{u}{x} + \pderiv{u}{y}+\pderiv{v}{x}+xy=0
\end{dcases} \\ 
$$ \\ 
$$L_{0}u = \begin{bmatrix}
2\pderiv{}{x}+3\pderiv{}{y} \ & \ 0\pderiv{}{x}-3\pderiv{}{y} \\ \\
\pderiv{}{x}+\pderiv{}{y}\ & \ \pderiv{}{x} + 0\pderiv{}{y} 
\end{bmatrix}\begin{bmatrix}
u \\ v
\end{bmatrix}$$ \\
Запишем характеристическую матрицу:
$$A = \begin{bmatrix}
2\xi_1+3\xi_2 \ & \ -3\xi_2 \\ \\
-\xi_1 + \xi_2 \  & \ \xi_1 
\end{bmatrix}$$
Характеристический многочлен:\\ \\
$P = 2\xi_1^2 + 3\xi_1\xi_2-3\xi_1\xi_2+3\xi_2^2=2\xi_1^2+3\xi_2^2$ \\ \\
$D = -2*6<0\rightarrow $ эллиптический.

\end{document}
