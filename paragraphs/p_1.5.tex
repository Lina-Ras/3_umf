\makeatletter
\makeatother
\documentclass[../main.tex]{subfiles}

\graphicspath{
    {img}
	{../img/}
}
\begin{document}
\section{Системы дифференциальных уравнений с частными производными.}
Рассмотрим $k$ функций:
$$u_1(x_1,...,x_n),u_2(x_1,...,x_n),...,u_k(x_1,...,x_n),$$ \\
и рассмотрим систему из k уравнений: 
$$\begin{dcases}
F_1(x_1, x_2,..., x_n, z_1, z_2,..., z_{N_1}), \\
F_2(x_1, x_2,..., x_n, z_1, z_2,..., z_{N_2}), \\
......................................... \\
F_k(x_1, x_2,..., x_n, z_1, z_2,..., z_{N_k})
\end{dcases} $$ \\
Имеем:
$$\begin{dcases}
F_1(x_1, x_2,..., x_n, \pderiv{u_1}{x_1},...,\pderiv{u_k}{x_n},..., \frac{\partial ^{m_1} u_k}{\partial x^{m_1}_n}) = 0, \\
F_2(x_1, x_2,..., x_n, \pderiv{u_1}{x_1},...,\pderiv{u_k}{x_n},..., \frac{\partial ^{m_2} u_k}{\partial x^{m_2}_n}) = 0, \\
........................................................ \\
F_1(x_1, x_2,..., x_n, \pderiv{u_1}{x_1},...,\pderiv{u_k}{x_n},..., \frac{\partial ^{m_k} u_k}{\partial x^{m_k}_n}) = 0, \\
\end{dcases} $$ \\
Записанная выше система является системой ДУ с частными производными.\\
Рассмотрим линейную систему 1-го порядка относительно двух функций:\\
$$
\begin{dcases}
b_{11}\pderiv{u}{x} + b_{12}\pderiv{u}{y} + c_{11}\pderiv{v}{x} + c_{12}\pderiv{v}{y}+b_1u + c_1v = f_1 \\ 
b_{21}\pderiv{u}{x} + b_{22}\pderiv{u}{y} + c_{21}\pderiv{v}{x} + c_{22}\pderiv{v}{y}+b_2u + c_2v = f_2
\end{dcases}$$\\
Представим данную систему в матричном виде:\\
$$Lu = f,$$ \\
где:
$$
L = \begin{bmatrix}
c_{11}\pderiv{}{x} + c_{12}\pderiv{}{y} + c_1 \ & \ b_{11}\pderiv{}{x} + b_{12}\pderiv{}{y} + b_1 \\ \\ 
c_{21}\pderiv{}{x} + c_{22}\pderiv{}{y} + c_2 \ & \ b_{21}\pderiv{}{x} + b_{22}\pderiv{}{y} + b_2
\end{bmatrix}
$$ \\ 
$$u = \begin{bmatrix}
u \\ v
\end{bmatrix}, \  f = \begin{bmatrix}
f_1 \\ f_2
\end{bmatrix}$$ \\
Выделим главную часть: \\ 
$$
L_0 = \begin{bmatrix}
c_{11}\pderiv{}{x} + c_{12}\pderiv{}{y} \ & \ b_{11}\pderiv{}{x} + b_{12}\pderiv{}{y} \\ \\ 
c_{21}\pderiv{}{x} + c_{22}\pderiv{}{y} \ & \ b_{21}\pderiv{}{x} + b_{22}\pderiv{}{y} 
\end{bmatrix}
$$ \\ \\
Рассмортим $L_0u$ и проведем замену : $\pderiv{}{x} \rightarrow \xi_1, \pderiv{}{y} \rightarrow \xi_2$. \\
\\ Запишем характеристическую матрицу
$$
A = \begin{bmatrix}
c_{11}\xi_1 + c_{12}\xi_2 \ & \ b_{11}\xi_1 + b_{12}\xi_2 \\ \\ 
c_{21}\xi_1 + c_{22}\xi_2 \ & \ b_{21}\xi_1 + b_{22}\xi_2 
\end{bmatrix},
$$ \\ \\
и характеристический многочлен:\\
$$P(\xi) = detA = a_{11}\xi_1^2 + 2a_{12}\xi_1\xi_2+a_{22}\xi_2^2,$$
где
$$a_{11} = c_{11}b_{21} - c_{21}b_{11}, \ \ 
a_{22} = c_{12}b_{22} - c_{22}b_{12}$$
$$a_{12} = \frac{1}{2}(c_{11}b_{22} + c_{12}b_{21} - c_{21}b_{12} - c_{22}b_{11})$$ \\ 
Классификация системы ДУ 1-го порядка для двух функций проводится таким же образом, как и классификация дифференциального уравнения 2-го порядка:
\\$$D = a_{12}^2 - a_{11}a_{22} $$ \\
$D > 0$ - гиперболический тип \\
$D = 0$ - параболический тип \\
$D < 0$ - эллиптический тип \\
\\
\\
\begin{example}
    Определить тип системы
    $$
    \begin{dcases}
    2\pderiv{u}{x} + 3\pderiv{u}{y}-3\pderiv{v}{y}+u=0 \\
    -\pderiv{u}{x} + \pderiv{u}{y}+\pderiv{v}{x}+xy=0
    \end{dcases} \\ 
    $$ \\ 
    $$L_{0}u = \begin{bmatrix}
    2\pderiv{}{x}+3\pderiv{}{y} \ & \ 0\pderiv{}{x}-3\pderiv{}{y} \\\\
    \pderiv{}{x}+\pderiv{}{y}\ & \ \pderiv{}{x} + 0\pderiv{}{y} 
    \end{bmatrix}\begin{bmatrix}
    u \\\\ v
    \end{bmatrix}$$ \\
    Запишем характеристическую матрицу:
    $$A = \begin{bmatrix}
    2\xi_1+3\xi_2 \ & \ -3\xi_2 \\ \\
    -\xi_1 + \xi_2 \  & \ \xi_1 
    \end{bmatrix}$$
    Характеристический многочлен:\\ \\
    $P = 2\xi_1^2 + 3\xi_1\xi_2-3\xi_1\xi_2+3\xi_2^2=2\xi_1^2+3\xi_2^2$ \\ \\
    $D = -2\cdot6<0\rightarrow $ эллиптический.
\end{example}
\end{document}
