\makeatletter
\makeatother
\documentclass[../main.tex]{subfiles}

\graphicspath{
    {img}
	{../img/}
}

\begin{document}
\section{Решение первой смешанной задачи для волнового уравнения
  методом разделения переменных}
 %Глава 2 Параграф 8 па Казлоўскай
\textit{Кратко(под удаление, подробный параграф про смешанные задачи будет позже):\\
	Смешанные условия $==$ начальные условия $+$ граничные условия\\
	Перванная смешанная задача $==$ граничные условия являются функциями\\}

Решим однородное волновое уравнение:

\begin{gather}
	\label{eq:2.8.1}
	\pderivt{u}{t} = a^2 \pderivt{u}{x}, \quad 0<x<l \quad t>0,
\end{gather}

Начальные условия:
\begin{gather}
	\label{eq:2.8.2}
	u|_{t=0}=\varphi(x), \quad \pderiv{u}{t}|_{t=0}=\psi(x)
\end{gather}

Граничные условия первого типа:
\begin{gather}
	\label{eq:2.8.3}
	u|_{x=0}=0,\quad u|_{x=l}=0, \quad t\geq 0
\end{gather}

Введём пространства:
\begin{gather*}
	V_1=\{\varphi \in C^2(0;l); \varphi'''\in L_2[0;l];
	\varphi''(0) = \varphi''(l) = \varphi(0) = \varphi(l) = 0\}\\
	V_2=\{\psi \in C^1(0;l); \psi''\in L_2[0;l];
	\psi(0) = \psi(l) = 0\}\\
	V = \{u \in C^2(0<x<l);\quad t>0\}
\end{gather*}

\subsection{Метод разделения переменных}
Будем искать решение в виде:
\begin{gather*}
	u(x,t) = X(x)\cdot T(t), \quad \text{где } X(x)\not\equiv 0, T(t)\not\equiv 0\\
	XT''=a^2TX'',  \qquad |/a^2TX\\
	\frac{T''}{a^2T} = \frac{X''}{X} = -\lambda^2 = const\\
\end{gather*}
Получим два ДУ:
\begin{numcases}{}
	T''+\lambda^2a^2T=0, \label{eq:2.8.4}\\
	X''+\lambda^2X=0 \label{eq:2.8.5}
\end{numcases}\\
Подставим в граничные условия (\ref*{eq:2.8.3}):
\begin{gather*}
	u|_{x=0}=X(0)\cdot T(t)=0\\
	\text{Из того, что } T(t)\not \equiv 0 \Rightarrow X(0)=0\\
	u|_{x=l}=X(l)\cdot T(t)=0 \Rightarrow X(l)=0
\end{gather*}
Тогда получим:
\begin{gather}
	\begin{cases} \label{eq:2.8.6}
		X''+\lambda^2X=0 \\
		X(0)=0           \\
		X(l)=0           \\
	\end{cases}
\end{gather}
(\ref*{eq:2.8.6}) -- \textbf{задача Штурма-Лиувилля}, или задача о
нахождении собственных функций $X$ и собственных значений $\lambda$.\\
Заметим, что:
\begin{enumerate}
	\item Собственные значения $\lambda \geq 0$
	\item Собственные функции ортогональны
	\item Собственные функции определены с точностью до постоянного множества
	\item Всякую функцию можно разложить в ряд Фурье по собственным функциям,
	      тк собственные функции ортогональны
\end{enumerate}
Запишем характерестическое уравнение для (\ref*{eq:2.8.6}):
\begin{gather*}
	\begin{array}[2]{c}
		\nu^2+\lambda^2=0 \\
		\nu^2 = -\lambda^2
	\end{array}
	\Rightarrow X(x)=A\cos(\lambda x)+B\sin(\lambda x)
\end{gather*}
Подставим полученное в условия задачи Штурма-Лиувилля:
\begin{gather*}
	X(0) = A = 0\\
	X(l) = B\sin(\lambda l)=0
\end{gather*}
Положим $B=1$. Тогда:
\begin{gather*}
	\lambda_n l=\pi n, \quad n=1,2\dots\\
	\lambda_n = \frac{\pi n}{l}, \quad n=1,2\dots
\end{gather*}
Решим (\ref{eq:2.8.5}):
\begin{gather*}
	T_n''+\left(\frac{\pi n a}{l}\right)^2T=0 \\ \Rightarrow
	T_n(t) = C_n\cos\left(\frac{\pi n a}{l}t\right) +  D_n\sin\left(\frac{\pi n a}{l}t\right)
\end{gather*}
Можем построить множество \textit{частных} решений:
\begin{gather*}
	U_n(x,t)=X_n\cdot T_n=\left(C_n\cos{\frac{\pi n a}{l}t} +D_n\sin{\frac{\pi n a}{l}t}\right)
	\sin{\frac{\pi n}{l}x}, \quad n=1,2\dots
\end{gather*}
Так как исходное уравнение линейное и однородное, то сумма частных решений будет являтся решением исходного.
\[
	u(x,t) = \sum_{n=1}^{\infty}X_n\cdot T_n =
	\sum_{n=1}^{\infty} \left(C_n\cos{\frac{\pi n a}{l}t} +D_n\sin{\frac{\pi n a}{l}t}\right)\sin{\frac{\pi n}{l}x}
\]
Из начальных условий (\ref{eq:2.8.2}) найдём $C_n$ и $D_n$:
\begin{gather*}
	u|_{t=0} = \sum_{n=1}^{\infty}C_n\sin{\frac{\pi n}{l}x} =
	\varphi(x) = [\text{ряд Фурье}] = \sum_{n=1}^{\infty}\varphi_n\sin{\frac{\pi n}{l}x},\\
	\text{где } \varphi_n = \frac{2}{l}\int\displaylimits_0^l \varphi(x)\sin\left({\frac{\pi n}{l}x}\right)dx\\
	C_n=\varphi_n
\end{gather*}
\begin{gather*}
	\pderiv{u}{t}|_{t=0} = \sum_{n=1}^{\infty}\frac{\pi n a}{l}D_n\sin{\frac{\pi n}{l}x} =
	\psi(x) = [\text{ряд Фурье}] = \sum_{n=1}^{\infty}\psi_n\sin{\frac{\pi n}{l}x},\\
	\text{где } \psi_n = \frac{2}{l}\int\displaylimits_0^l \psi(x)\sin\left({\frac{\pi n}{l}x}\right)dx\\
	D_n=\frac{l}{\pi n a}\psi_n
\end{gather*}
Для поиска собственных функций воспользуемся изученным на курсе функционального анализа:
\begin{gather*}
	\varphi(X) = \sum_{n=1}^{\infty} \varphi_n X_n\\
	(\varphi, X_m) = \sum_{n=1}^{\infty} \varphi_n (X_n,X_m)\\
	\forall n \neq m:\quad (X_n,X_m) = 0 \qquad n=m: \quad (X_n,X_m)=||X_m||^2
\end{gather*}
Отсюда следует, что:
\begin{gather*}
	(\varphi, X_m) = \varphi_m\cdot ||X_m||^2 \qquad \Rightarrow  \qquad \varphi_m = \frac{(\varphi, x_m)}{||x_m||^2}\\
	\varphi = \frac{
		\int\displaylimits_0^l \varphi(x)X_m(x)dx
	}{
		\int\displaylimits_0^l X_m^2(x)dx
	}
\end{gather*}
\end{document}