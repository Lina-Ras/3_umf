\makeatletter
\makeatother
\documentclass[../main.tex]{subfiles}

\graphicspath{
    {img}
	{../img/}
}
\begin{document}
\section{Замена независимых переменных в уравнениях 2-го порядка с двумя независимыми переменными.}
Запишем уравнение:
\begin{equation}\label{eq:1.6.1}
	a_{11}(x,y)\frac{\partial^2u}{x^2} + 2a_{12}(x,y)\frac{\partial^2u}{\partial x \partial y} + a_{22}(x,y)\frac{\partial^2u}{y^2} + a(x,y)\pderiv{u}{x} +b(x,y)\pderiv{u}{y}+c(x,y) = f(x,y) \end{equation} \\
Рассмотрим этo уравнение в области $\Omega \subset \R^2$, \ \ $a_{ij} \in C^2(\Omega)$ \\
\\ Введём замену:
$$\begin{dcases}\xi = \phi(x,y) \\ \eta = \psi(x,y)\end{dcases}$$
\\ \\
$$\begin{vmatrix}
		\phi_x \  & \ \phi_y
		\\
		\psi_x \  & \ \psi_y
	\end{vmatrix} \ne 0, \ \ \phi(x,y), \psi(x,y) \ \in C^2(\Omega)
$$ \\
$
	u_x = u_\xi\xi_{x}+u_\eta\eta_x
$  \\$
u_{xx} = u_{\xi\xi}(\xi_x)^2 + 2u_{\xi\eta}\xi_x\eta_x+u_{\eta\eta}(\eta_x^2)+u_\xi\xi_{xx}+u_\eta\eta_{xx}
$\\ $
u_y=u_\xi\xi_y+u_\eta\eta_y
$\\
$u_{yy}=u_{\xi\xi}(\xi_y)^2+2u_{\xi\eta}\eta_y+u_{\eta\eta}(\eta_y)^2+u_\xi\xi_{yy}+u_\eta\eta_{yy}$\\
$u_{xy}=u_{\xi\xi}\xi_{x}\xi_y+u_{\xi\eta}(\xi_x\eta_y+\xi_y\eta_x)+u_{\eta\eta}\eta_x\eta_y+u_\xi\xi_{xy}+u_\eta\eta_{xy}$ \\
	\begin{equation}
		\overline{a_{11}}u_{\xi\xi}+2\overline{a_{12}}u_{\xi\eta}+\overline{a_{22}}u_{\eta\eta}+\overline{a}u_{\xi}+\overline{b}u_\eta+cu=f
	\end{equation}
	\begin{equation}
		\begin{dcases}
			\overline{a_{11}}=a_{11}\xi_x^2+2a_{12}\xi_{12}\xi_x\xi_y+a_{22}(\xi_y)^2
			\\
			\overline{a_{22}}=a_{11}\eta_x^2+2a_{12}\eta_x\eta_y+a_{22}(\eta_y)^2
			\\
			\overline{a_{12}}=a_{11}\xi_x\eta_x+a_{12}(\xi_x\eta_y+\xi_y\eta_x)+a_{22}\xi_y\eta_y
		\end{dcases}
	\end{equation}
	Приведём к каноническому виду:
	\begin{equation}\label{1.6.4}
		a_{11}(\pderiv{z}{x})^2+2a_{12}\pderiv{z}{x}\pderiv{z}{y}+a_{22}(\pderiv{z}{y})^2=0
	\end{equation}
	Решим квадратное уравнение:
	\begin{equation}\label{1.6.5}
		a_{11}(dy)^2-2a_{12}dydx+a_{22}(dx)^2=0
	\end{equation}
	\begin{equation}\label{1.6.6}
		\begin{dcases}
			\frac{dy}{dx} = \frac{a_{12}+\sqrt{a_{12}^2-a_{11}a_{22}}}{a_{11}}=\lambda_1(x,y) \\
			\frac{dy}{dx} = \frac{a_{12}-\sqrt{a_{12}^2-a_{11}a_{22}}}{a_{11}}=\lambda_2(x,y)
		\end{dcases}
	\end{equation}
	\textbf{Теорема:}\\
	\par Если $\phi(x,y)$ является первым интегралом (\ref{1.6.6}), то $z=\phi(x,y)$ - является решением уравнения (\ref{1.6.4}). \\ \\
	\textit{Доказательство:}\\
	\par Если $\phi(x,y)$ является первым интегралом (\ref{1.6.6}), то $\phi(x,y) \in C^1(\Omega), grad\phi \ne 0, \phi(x,y(x))=C$. Далее для фиксированного $M_0\in\Omega$ строим решение $y(x)$ уравнения $\frac{dy}{dx}=\lambda_1$, удовлетворяющего условию $y(x_0)=y_0$.\\
	\par По теореме Пикара-Линделёфа, такое решение существует в окрестности $x_0$. Тогда $\phi(x,y(x))=C_0$
	$$\pderiv{\phi}{x}+\pderiv{\phi}{y}\pderiv{y}{x}|_{y=y(x)}=0$$
	$$\pderiv{\phi}{x}+\pderiv{\phi}{y}\pderiv{y}{x}|_{M_0}=0$$
	Если в этой точке $\pderiv{\phi}{y}=0$, то $\pderiv{\phi}{x}=0$, а это противоречит условию 1-го интеграла. Значит в точке $M_0$:
	$$\frac{dy}{dx}=-\frac{\pderiv{\phi}{x}}{\pderiv{\phi}{y}}$$
	Из (\ref{1.6.5}):
	$$ a_{11}(\frac{dy}{dx})^2-2a_{12}\frac{dy}{dx}+a_{22}=0$$
	$$ a_{11}(\frac{\pderiv{\phi}{x}}{\pderiv{\phi}{y}})^2-2a_{12}\pderiv{\phi}{x}\pderiv{y}{\phi}+a_{22}=0$$
	$$ a_{11}(\pderiv{\phi}{x})^2+2a_{12}\pderiv{\phi}{x}\pderiv{y}{\phi}+a_{22}(\pderiv{\phi}{x})^2=0$$
	Результаты соответствуют (\ref{1.6.4}), а так как $z = \phi(x,y)$, это и будет решением (\ref{1.6.4}).
\end{document}
