\makeatletter
\makeatother
\documentclass[../main.tex]{subfiles}

\graphicspath{
    {img}
	{../img/}
}

\begin{document}
\section{Корректность задачи Коши для волнового уравнения}
\begin{large}
    \begin{center}
        $\pderivt{u}{t}-a^2\pderivt{u}{x}=0$, $\D=\{ -\infty<x<+\infty, |t|\leq T\rbrace$\\
        $ $\\
        $ u|_{t=0} = \varphi(x)$, $-\infty<x<+\infty$\\ 
        $ $\\
        $\pderiv{u}{t}|_{t=0} = \psi(x),  -\infty<x<+\infty$\\
        $ $\\
        $u(x,t)=\frac{\varphi(x-at)+\varphi(x+at)}{2}+\frac{1}{2a}\int_{x-at}^{x+at}\psi(\xi)d\xi$
    \end{center}
\end{large}
Введём метрики для \[V_1:p_1(\varphi_1,\varphi_2)=\sup_{-\infty<x<+\infty}|\varphi_1-\varphi_2|,\] \[V_2:p_2(\psi_1,\psi_2)=\sup_{-\infty<x<+\infty}|\psi_1-\psi_2|\]\[V:p(u_1,u_2)=\sup_{-\infty<x<+\infty}|u_1-u_2|\]\[ |t| \leq T\]\\
Само выведение функции Даламбера позволяет утверждать, что решение задачи существует и единственно.\\\\
    Рассмотрим 2 задания:\\\\
$
\begin{large}
\begin{left}
\lbrace
\begin{array}{cc}
     \pderivt{u_i}{t}=a^2\pderivt{u_i}{x}\\\\
     u_i|_{t=0}=\varphi_i(x)\\\\
     \pderiv{u_i}{t}|_{t=0}=\psi_i(x)
\end{array}
\right.
\end{left}  
\end{large}
$
$i=\overline{1,2}$\\ \\
Надо показать, что $\forall \epsilon > 0$  $\exists \delta > 0$,  что как только $p_1(\varphi_1, \varphi_2)<\delta$, $p_2(\psi_1, \psi_2)<\delta$, то $p(u_1, u_2)<\epsilon$\\
Оценим\\
$|u_1-u_2| \leq |\frac{1}{2}\varphi_1(x+at)-\frac{1}{2}\varphi_2(x+at)|+|\frac{1}{2}\varphi_1(x-at)-\frac{1}{2}\varphi_2(x-at)|+\frac{1}{2a}\int_{x-at}^{x+at}|\psi_1(\xi)-\psi_2(\xi)|d\xi$<\\\\$\frac{\delta}{2}+\frac{\delta}{2}+\frac{\delta}{2a}\int_{x-at}^{x+at}d\xi \leq \delta+\delta|t| \leq \delta(1+T)$, а следовательно, если выбрать  $0<\delta<\frac{\xi}{1+T}$, то получим $p(u_1, u_2)<\xi$.\\\\
Таким образом показали, что задача устойчива, значит задача Коши для волнового уравнения поставлена корректно.
\end{document}
